\documentclass[conference]{IEEEtran}
\IEEEoverridecommandlockouts
% The preceding line is only needed to identify funding in the first footnote. If that is unneeded, please comment it out.
\usepackage{cite}
\usepackage{amsmath,amssymb,amsfonts}
\usepackage{algorithmic}
\usepackage{graphicx}
\usepackage{textcomp}
\usepackage{xcolor}
\def\BibTeX{{\rm B\kern-.05em{\sc i\kern-.025em b}\kern-.08em
    T\kern-.1667em\lower.7ex\hbox{E}\kern-.125emX}}
\begin{document}

\title{Visualizing Software Evolution in Linux: A Hierarchical Graph-Based Heat Map Approach\\

\thanks{}
}

\author{\IEEEauthorblockN{Adrian Volpe}
\IEEEauthorblockA{\textit{College of Science and Math} \\
\textit{Belmont University}\\
Nashville, USA \\
adrianvolpemath@gmail.com}
\and
\IEEEauthorblockN{Esteban Parra}
\IEEEauthorblockA{\textit{College of Science and Math} \\
\textit{Belmont University}\\
Nashville, USA \\
esteban.parrarodriguez@belmont.edu}}

% Try using this alternative long format instead if modifying the original leads to it looking weird
% \author{\IEEEauthorblockN{Michael Shell\IEEEauthorre
% fmark{1}, Homer Simpson\IEEEauthorrefmark{2}, James K
% irk\IEEEauthorrefmark{3}, Montgomery Scott\IEEEautho
% rrefmark{3} and Eldon Tyrell\IEEEauthorrefmark{4}}
% \IEEEauthorblockA{\IEEEauthorrefmark{1}School of Ele
% ctrical and Computer Engineering\\
% Georgia Institute of Technology, Atlanta, Georgia 30
% 332--0250\\
% Email: mshell@ece.gatech.edu}
% \IEEEauthorblockA{\IEEEauthorrefmark{2}Twentieth Cen
% tury Fox, Springfield, USA\\
% Email: homer@thesimpsons.com}


\maketitle

\begin{abstract}
This project presents a visualization of Linux commit activity using a hierarchical, heat-map based graph. Using a large dataset of commit data from Zenodo, we model the Linux structure as a directed hierarchy. Each node in the graph represents a subsystem, and edges show a parent-child relationship within the Linux architecture. The graph is rendered radially in Unity with a color gradient applied to nodes to indicate the volume of commits made to each subsystem. The result is an interactive, intuitive view of development hotspots in the Linux kernel, aimed at supporting further software evolution analysis. 
\end{abstract}

\begin{IEEEkeywords}
commit, graph, heat-map
\end{IEEEkeywords}

\section{Introduction}
Linux is a family of open source operating systems that are based on the Linux kernel. The Linux kernel is a core component of a device that manages the hardware and resources. It's responsible for the CPU, memory and peripherals. Linux is versatile and is used everywhere from powering smartphones to operating smart TV's. Linux is fast, secure and highly scaleable, so it is important to understand how it has changed overtime.% talk about what a linux is and what it does and why its useful (free, safe, fast)

Linux is an open source system, meaning that the source code is available, for free, to be modified and redistributed. However, the main Linux operating system (OS) is only modified by specific authors and each modification is checked thoroughly. These changes to the Linux OS source code are called commits. Each commit contains certain data including: the author of the commit, the time the commit was made, every line changed, and the number of added and deleted lines. Commits are used to keep track of what code was changed, when it was changed and by who. % what is linux/ what is linux used for / what are commits / what are commits used for /  


Analyzing commit history is useful because it can show areas in the Linux kernel that have not recently been modified or that have been modified heavily. Areas that have recently been modified often are more likely to contain bugs or errors. Thus commit history visualization can strengthen our ability to find bugs in the Linux OS. We created a graph visualization of Linux subsystems using commit frequency as a heat metric. This visualization is specifically useful because it provides an interactive and clear way of seeing commit history. This approach has not been done before and can provide a new way of seeing the relationship between the Linux OS structure and how commits are distributed across it. % why is this useful/ why is looking at commit activity useful / how is this different

We will discuss some related works, the methods used to create this graph, results, and areas of future improvement.




\section{Related Work}

% read papers, summarize each one individually, then look at every summary and answer these questions in the paper: what is known? what is unknown? what has been done? what are you doing? how are they different? or do it this way : GENERAL OVERVIEW WHAT BEEN DONE, go through each paper and explain, GENERAL OVERVIEW WHA+Y DIFFERENT.

I have added something to the LaTeX file, now will it show up on GitHub? Im gonna add even more chnages like this one here. Let me make a change here and see where it appears when I hit save.

\section{Methodology}

\subsection{Data Collection}
The data we used came from the Zenodo dataset provided by VISSOFT. We created code in Java to read the JSON file provided and extract data about the commits. Data extracted includes the name of the subsystem, and how many commits were made to that subsystem. The Java code output a .txt file listing out the subsystem name, number of commits made, and corresponding color.

\subsection{Graph Construction}
To construct the graph we turned each subsystem into a node and connected two nodes if and only if one subsystem is a child of another. Thus our graph will show the parent-child relationship between all of the subsystems. Each node is assigned a color such that the more commits made to that subsystem, the more red the node will appear and fewer commits are a green color.

\subsection{Color Encoding}
At first, all nodes took on an indistinguishable shade of green, except for a select few nodes. This was because the range of commits made to subsystems was huge. The maximum number of commits made to a subsystem was 500,000 while the minimum was 0. The average was somewhere around 1000 commits per subsystem. Thus, when we colored each node by taking a ratio between each subsystems number of commits by the maximum number of commits it heavily skewed the nodes to be green. To fix this we implemented a logarithmic scale. The color of each node is now determined by the log of that previous ratio, except in cases where the ratio becomes $0$ or $1$. Those cases are handled individually, since $\log(0)$ is undefined and $log(1)$ returns a green color which is the opposite of what is desired when the ratio is $1$. This logarithmic scaling fixed the issue with most nodes looking identical. Nodes now take on a range of colors from green to red.

\subsection{Implementation}
The implementation of this code in Unity is interactive. The user is able to pan the camera, zoom in/out and interact with the nodes. When the user clicks on a node information about that node appears, including the name of the subsystem and the exact number of commits made to that subsystem.

\section{Results}

The preferred spelling of the word ``acknowledgment'' in America is without 
an ``e'' after the ``g''. Avoid the stilted expression ``one of us (R. B. 
G.) thanks $\ldots$''. Instead, try ``R. B. G. thanks$\ldots$''. Put sponsor 
acknowledgments in the unnumbered footnote on the first page.

\section{Discussion}

Please number citations consecutively within brackets \cite{b1}. The 
sentence punctuation follows the bracket \cite{b2}. Refer simply to the reference 
number, as in \cite{b3}---do not use ``Ref. \cite{b3}'' or ``reference \cite{b3}'' except at 
the beginning of a sentence: ``Reference \cite{b3} was the first $\ldots$''

Number footnotes separately in superscripts. Place the actual footnote at 
the bottom of the column in which it was cited. Do not put footnotes in the 
abstract or reference list. Use letters for table footnotes.

Unless there are six authors or more give all authors' names; do not use 
``et al.''. Papers that have not been published, even if they have been 
submitted for publication, should be cited as ``unpublished'' \cite{b4}. Papers 
that have been accepted for publication should be cited as ``in press'' \cite{b5}. 
Capitalize only the first word in a paper title, except for proper nouns and 
element symbols.

For papers published in translation journals, please give the English 
citation first, followed by the original foreign-language citation \cite{b6}.

\section{Conclusion}


\section{References}



\begin{thebibliography}{00}
\bibitem{b1} G. Eason, B. Noble, and I. N. Sneddon, ``On certain integrals of Lipschitz-Hankel type involving products of Bessel functions,'' Phil. Trans. Roy. Soc. London, vol. A247, pp. 529--551, April 1955.
\bibitem{b2} J. Clerk Maxwell, A Treatise on Electricity and Magnetism, 3rd ed., vol. 2. Oxford: Clarendon, 1892, pp.68--73.
\bibitem{b3} I. S. Jacobs and C. P. Bean, ``Fine particles, thin films and exchange anisotropy,'' in Magnetism, vol. III, G. T. Rado and H. Suhl, Eds. New York: Academic, 1963, pp. 271--350.
\bibitem{b4} K. Elissa, ``Title of paper if known,'' unpublished.
\bibitem{b5} R. Nicole, ``Title of paper with only first word capitalized,'' J. Name Stand. Abbrev., in press.
\bibitem{b6} Y. Yorozu, M. Hirano, K. Oka, and Y. Tagawa, ``Electron spectroscopy studies on magneto-optical media and plastic substrate interface,'' IEEE Transl. J. Magn. Japan, vol. 2, pp. 740--741, August 1987 [Digests 9th Annual Conf. Magnetics Japan, p. 301, 1982].
\bibitem{b7} M. Young, The Technical Writer's Handbook. Mill Valley, CA: University Science, 1989.
\end{thebibliography}
\vspace{12pt}
\color{red}
IEEE conference templates contain guidance text for composing and formatting conference papers. Please ensure that all template text is removed from your conference paper prior to submission to the conference. Failure to remove the template text from your paper may result in your paper not being published.

\end{document}